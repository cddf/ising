\section{Das Ising-Modell}
Das Ising-Modell ist eine Vereinfachung des Heisenberg-Modells. Die Spins werden nicht als Vektor behandelt, sondern auf $s_i^z = \pm 1$ reduziert.
\begin{equation}
  \hat{\mathcal H} = -\frac 12\sum _{ij} J_{ij} S_i^z S_j^z - B_z \sum_{i=1}^N S_i^z
\end{equation}
%TODO schöne Einleitung
Deshalb ist das Ising-Modell vor allem dann eine gute Näherung des Heisenberg-Modells, wenn sich durch Anisotropien eine vorgezogene Richtung ergibt, oder wenn sich die Spins an einem Isotropen äußeren Magnetfeld ausrichten.
Um das Modell weiter zu vereinfachen, wird in der Regel für $J_{ij}$ eine nächste-Nachbar-Wechselwirkung angenommen. $J$ ist konstant für das gesamte System und es wird nur die Wechselwirkung mit den direkten Nachbarn berechnet.

Mit dem Modell kann der Phasenübergang eines Ferromagneten in zwei und mehr Dimensionen beschrieben werden.